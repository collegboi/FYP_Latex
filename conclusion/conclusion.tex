\chapter{Conclusion}

\label{ch:conclusions}

\section{Introduction}

In this conclusion chapter, the project plan, future work, strengths and weaknesses of the project will be discussed. Reflective and learning outcomes will also be included.

\section{Project plan}

The objective plan for this project was to design and develop a way to improve mobile applications in development, testing, production and can also include acceptance (DTAP). The project has changed from the beginning within a few months into it. It started out with developing the back-end using Django framework and Python for the programming language. This changed to using Perfect ( Swift server side ) for the back-end and meaning that Swift was used for both client and server. The reasons behind this because the project will be open sourced, it will allow developers are already developing their apps in Swift to contribute towards it. Having software open source has its benefit which not only include having developers contributing and in return having free software. It can also help with providing more functionality, services and security to the system.

The initial plan for the dashboard was to develop it for an iPad application but changed after some thought and discussions with other developers. The dashboard was designed and developed for Mac app, and the main reason for this change was to not restrict developers from having to have an iPad. Professional developers in a company could have access to an iPad, but as this project is trying to reach new developers, then this would have been an issue.

One of the parts of the remote configuration initially was to be able to move the objects in each view, so where a label or text field is positioned. After having a discussion with a professional mobile developer, this part was put aside. His reasoning was that constraints which are how the UI objects are positioned together are already a fully integrated set-up.

\section{Future work}

Plans for the future include redesigning the dashboard interface, due to time this could not have been done. In the services section, each view requires the user to choose the particular mobile application and version every-time the view is opened. This idea behind this was to able to stay on one screen and complete the tasks for every version necessary, and the equivalent on the other screens. It can disrupt the flow when having to keep changing the application, and also mistakes can be made when jumping through different versions.

Another plan is to get more feedback from developers as they begin to use the system, to see where the project can go. The remote configuration feature can be expanded to more UI objects within the application, and due to time, was restricted to a handful. There more features that can be added to give the developers more tools to maintain their apps. Some of these include a live database, OAuth. The live database creates a continuous connection between the end users and their data, so not having to keep refreshing a view for updates. OAuth is an extra security feature that can be included to keep usernames and password and other information private. OAuth uses other services such as Facebook, and Google to log-in and is authenticated through their services.


The next big plan is to develop a framework for Android applications, the type that was designed in this project for iOS. By doing this can broaden the scope that this system can reach and be integrated into mobile applications. The web-server is already configured to handle request from different mediums, as it uses a common protocol called an API. The remote configuration which defines the properties of the UI objects will need some configuration to add the Android framework.

\section{Project Weaknesses}

One of the weaknesses of this project, as talked about in the plans, is the design of the dashboard. The additional requirement to keep choosing the mobile application name and version from the drop-down list in some of the views. Although this has some benefits of been able to stay on one view and update multiple of apps, this could cause issues with updating the wrong version or app. Another weakness of the project but is mainly due to time, is that the system is currently limited to iOS devices, but plans are in place to changed this. This weakness is one of the largest of this project, as the statics of the highest number of mobile apps is Android, so bringing this project to that area will increase the likelihood of the system being widely used.

\section{Project Strengths}

The strength of this project is that it proved that mobile application in developed could be improved, by reducing the amount of work required. The current way applications are designed can be changed, give end users freedom of how the looks and feels with some restriction. Unconsciously people act differently on the appearance of the apps, so been able to provide them with some rights to change can in turn potentially make them want to keep and use them. The configuration for setting up the web-server, and including the SDK in the apps is developed to make easy to use. Then to use the SDK and communicate with their web-server has the added benefit of simplicity.

\section{Learning outcomes}

% Thinking back to when I thought of the project, some of the features such as storing objects in the cloud database without having to write too much code, and remote configuration which has enabled apps to be updated in production would be challenging. To not only think that is would be a challenge but also if Apples strict pre-publish checks would allow applications to be updated when published has been a drive. 

Personal development has been with research, the amount of done with this project out ways any other project done before. Researching my project idea has shown me skills to when developing software always to think ten steps ahead of where possible. Instead of starting the development and studying along the way but start with the research has shown a whole new way to creating software. This way of building software has advantages such as finding the potential issues and risks early on to overcome. Speaking with professional mobile developers when I did the research has helped with the project with not only validating my idea that it has potential but also giving me constructive feedback. The feedback provided has made me do more research regarding other services that companies are doing to look into. One developer I had an interview with also pointed out some areas to be cautious about. The research gone into this project apart from technologies and methodologies being used has shown the need for a mobile backend as a service.


\section{Conclusion}
The primary aim of this project was to create a system that can bring more people to start developing mobile applications. To bring a new way of developing an application, and maintaining the application quicker with ease. After getting the project evaluated from experienced developers leading to a conclusion that this project has plausibility.