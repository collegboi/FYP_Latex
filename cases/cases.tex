\chapter{Cases}

\section{Research}

\subsection{Tapadoo}
Tapadoo \cite{tapadoo} gave me more constructive feedback, stating that while the remote configuration service on its own is good it has its drawbacks. A developer's point of view is that updating the user interface will not happen often enough to validate the use of it. However using the remote configuration along with A/B testing ( comparing two variations of an app against each other ) is a powerful, useful tool for developers and customers. Instead of relying on what the designer or the project manager thinks, they leave it up to the users by viewing the analytics based on the two different variations.

He also mentioned to be careful when using the translation files and allowing the users to choose their own language as this goes against Google’s and Apple’s guidelines. He stated that there are services already implemented called Localization that deals with the displaying the correct translation. He also explained a need for updating content within an app, changing the button title from “Pay” to “Pay Now” for example and that my project should include this service.

\subsection{Trust5}

Trust5 \cite{trust5} had two developers to meet with me, they expressed interest in the idea and said it is a powerful tool for developers to use. They gave me feedback to design it as a “white label product”, meaning that the product can be used by any company with their logo attached. We discussed the different features already implemented with regards the remote configuration and explained where to concentrate on and what to leave to last. The configuration is split up into three phases of testing as explained earlier, they talked about leaving the complicated part of converting user interface objects to Apple’s visual format language to last and just used the constraint object class until the majority of the project was completed.

\section{Evaluation}

\subsection{Trust5}

\subsubsection{Developer 1}

I found that the project was carried out in a professional manner and to a high standard. The concept was intriguing and well executed. Seeing the configuration changes being applied in a real-time and remote manner without need for client-side deployment is impressive. The fact that this remote configuration obviates the need for a client update to be released adds significant value as this is a time-consuming and painful process, in my experience this is particularly true for the Apple apps. I've no doubt that there are many real world scenarios that could leverage this functionality.

\subsubsection{Developer 2}

Library usability:

We’ve found the “MBaaSKit” library to be very easy to integrate with any iOS app. As a pod library it involved very little setup effort as it only takes one line of code to include the library and another one line command to import the library. Using Cocoapods and pod files is the most common practice to include dependencies/libraries.


Library capabilities:

We were really impressed with the capabilities of the “MBaaSKit” library. It is a really powerful tool that can save a lot of development time & effort especially on the design side of things. A developer can quickly create a mockup of any native view and subsequently change the look & feel of that view remotely. It can also help an app but hiding or disabling a button/feature dynamically.

Obviously, one can create an app that is essentially a HTML wrapper that displays content based on a HTML that is fetched remotely. However, that has two main disadvantages which are performance & security. With “MBaaSKit” it could give us the flexibility of changing content dynamically and without having to do a new release every time designers come up with a tweaked design. It can also help apps that are essentially the same but branded differently to different clients.

Also the library can be used as a functional test tool. The library can help load the app with some test configurations dynamically such as long text, non-latin characters, text alignment, etc..

For “MBaaSKit” to be useful; it also had to be a simple tool to use. Following the Sample project; it was evident that it was very simple to use as any UI element can be configured with only one line of code. 

Simplicity of the library is of utmost importance as we see the selling point of the tool, apart from providing flexibility & extending an app’s capability , is reducing development time.

General feedback (Improvements):

A nice feature to have is to be able to load different configurations based on a per user (or user type) level. For example an app developer might want to show or enable different areas of a view depending on the type of the user using the app (e.g. basic vs premium user). 

Improvements can be made to provide more documentation on the GitHub page on how to use the library with extra screenshots of the library’s capabilities. 

Also some guidance could be provided on how to use the library with an Objective-C project.

The sample project provides great examples on how to use the library and showcases its capabilities. However, we found the name of the sample project “MBaasKitTest” to be a little bit confusing as we weren’t sure initially if it was a test or a sample project.
