\chapter{Cases}

\section{Research}

\subsection{Tapadoo}
Tapadoo \cite{tapadoo} gave me more constructive feedback, stating that while the remote configuration service on its own is good it has its drawbacks. A developer's point of view is that updating the user interface will not happen often enough to validate the use of it. However using the remote configuration along with A/B testing ( comparing two variations of an app against each other ) is a powerful, useful tool for developers and customers. Instead of relying on what the designer or the project manager thinks, they leave it up to the users by viewing the analytics based on the two different variations.

He also mentioned to be careful when using the translation files and allowing the users to choose their own language as this goes against Google’s and Apple’s guidelines. He stated that there are services already implemented called Localization that deals with the displaying the correct translation. He also explained a need for updating content within an app, changing the button title from “Pay” to “Pay Now” for example and that my project should include this service.

\subsection{Trust5}

Trust5 \cite{trust5} had two developers to meet with me, they expressed interest in the idea and said it is a powerful tool for developers to use. They gave me feedback to design it as a “white label product”, meaning that the product can be used by any company with their logo attached. We discussed the different features already implemented with regards the remote configuration and explained where to concentrate on and what to leave to last. The configuration is split up into three phases of testing as explained earlier, they talked about leaving the complicated part of converting user interface objects to Apple’s visual format language to last and just used the constraint object class until the majority of the project was completed.

\section{Evaluation}

\subsection{Trust5}

I found that the project was carried out in a professional manner and to a high standard. The concept was intriguing and well executed. Seeing the configuration changes being applied in a real-time manner without need for client-side deployment is impressive. I've no doubt that there are many real world scenarios that could leverage this functionality.
